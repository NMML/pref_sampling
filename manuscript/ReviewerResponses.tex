
\documentclass[12pt,fleqn]{article}
%\documentclass[12pt,a4paper]{article}
\usepackage{amsmath,epsfig,epsf,psfrag,color,ecology,natbib}
%\usepackage{a4wide,amsmath,epsfig,epsf,psfrag}
\usepackage[nolists]{endfloat}
\bibliographystyle{ecology}

%  put your commands here

\def\be{{\ensuremath\mathbf{e}}}
\def\bx{{\ensuremath\mathbf{x}}}
\def\bftheta{{\ensuremath\boldsymbol{\theta}}}
\def\bfeta{{\ensuremath\boldsymbol{\eta}}}
\def\bfdelta{{\ensuremath\boldsymbol{\delta}}}
\def\bfnu{{\ensuremath\boldsymbol{\nu}}}
\def\bfmu{{\ensuremath\boldsymbol{\mu}}}
\def\bfSigma{{\ensuremath\boldsymbol{\Sigma}}}
\def\bfB{\textbf{B}}
\def\bfZ{\textbf{Z}}
\newcommand{\VS}{V\&S}
\newcommand{\tr}{\mbox{tr}}

\begin{document}

\bf \hspace{-.32in} Conn et al. responses to reviewer comments

%  make sure that the document has 25 lines per page (it is 12 pt)

\setlength{\textheight}{575pt} \setlength{\baselineskip}{23pt}

%  IMPORTANT -- place the \title, \author, etc, \begin{summary}...
%  \end{summary} \keywords statements in the order demonstrated below
\rm
Here, we provide line by line responses to associate editor and reviewer comments, suggestions, and edits.  For completeness, we include all reviewer comments in addition to our responses (our responses are in blue).  In addition, our revision includes copy editing corrections from our editorial office and some clarifications in response to additional comments we received from an interested reader. We provide a marked up PDF version of our manuscript showing all changes made.

In the process of conducting revisions we corrected an error in our estimation code (somewhere in the model development and testing process, the link function for $R_i$ was misspecified as multinomial-logit instead of logit).  Correcting this error has led to several qualitative changes to the paper.  Conclusions from the simulation study were strengthened; however, bearded seal conclusions were made more murky.  In particular, both of the model types incorporating preferential sampling (linear and trend surface) appeared to produce biologically unrealistic estimates in the bearded seal example.  This was particularly troubling because AIC favored these models.  We decided to conduct an additional cross validation analysis to quantify predictive ability, and found that non-preferential models produced more accurate predictions.  We now argue against using preferential sampling models for the case study.

\section{Associate editor}

[From decision letter]
In view of the comments received, I must decline your manuscript for publication in the Methods in Ecology and Evolution at this time.  However, we would be willing to consider a new manuscript which takes into consideration these comments. In addition to the other comments, I would ask you to try to make the presentation more accessible to an ecological audience - as you put it in the context of SDMs, I think it should really be understandable by ecologists working with SDMs. I appreciate that this can be difficult.

\textcolor{blue}{
We thank the associate editor for the invitation to resubmit, and for their constructive criticism.  In our revision, we have made an effort to make the manuscript more accessible to quantitative ecologists and to those working in the SDM modeling field.  We provide numerous examples below.}

To give one example, you wrote (p4, l78-80):
``For instance, PS can occur when the investigator uses a priori knowledge or observations of the state variable obtained during sampling to allocate survey effort in places where abundance or occurrence is known to be high."

which could be re-written as

``For instance, PS can occur when the investigator tends to sample more in places where a species has already been observed, or is known to be abundant."

\textcolor{blue} {This formulation is much more to the point; we have accepted this change.}

[From the comments to author section]
Your paper definitely focuses on an important issue, and provides ways to address the problem. Besides the issues raised by the referee, I think you should relate your paper a bit more to the ecological literature, in particular on SDMs (see Edwards et al. 2006). I think in particular that in addition to the types of sampling you discuss, one could also consider sampling designs that are optimal to estimate response curve (eg Albert et al. 2010 - sorry for the self-citation...). Ecological studies may have different objectives, and I think you should try to make your paper relevant to as many readers as possible.

Albert, C. H., N. G. Yoccoz, T. C. Edwards, C. H. Graham, N. E. Zimmermann, and W. Thuiller. 2010. Sampling in ecology and evolution � bridging the gap between theory and practice. Ecography 33:1028-1037.
Edwards, T. C., D. R. Cutler, N. E. Zimmermann, L. Geiser, and G. G. Moisen. 2006. Effects of sample survey design on the accuracy of classification tree models in species distribution models. Ecological Modelling 199:132-141.

\textcolor{blue}{We thank the editor for pointing us to these articles - our knowledge of the species distribution modeling literature is admittedly a little lacking.  These papers are certainly germane to the present study, and we have included additional material in the introduction (2nd paragraph) that talks about (1) implications of design vs. model-based inference, and (2) the importance of the particular objective function (e.g., minimizing variance, estimating species-habitat relationships) for optimal design.}

\textcolor{blue}{
Additionally, we have made the following changes to try to make the manuscript accessible to a wider audience:
\begin{itemize}
  \item An additional figure (fig. 1) to more clearly depict preferential sampling for visual learners.
  \item  In the methods section, we motivate inclusion of spatially autocorrelated random effects by indicating ``Spatially autocorrelated random effects are often included to allow animal density to vary smoothly in space and to account for residual patchiness not explained by predictive covariates."
  \item We changed ``it is usually impossible to view system state perfectly even in locations where sampling is conducted" to ``it is usually impossible to conduct a complete census of all animals present in a surveyed location"
  \item We provide examples of survey and observer specific covariates. (visibility, observer experience)
  \item We changed the ``underlying spatial field of interest" and ``spatially-referenced random field values" to ``the process of interest"
  \item We changed ``surveyed locations were selected independently of the abundance generating process" to ``surveyed locations were randomly selected (i.e. independent of abundance)."
  \item We changed ``In this case, the abundance process and sampling intensity submodels were independent, as is the with case canonical SDMs (at least when fitted to presence-absence or count data)" to ``In this case, sampling intensity is assumed to be independent of underlying abundance, as implicitly assumed by most SDMs in the ecological literature."
  \item Numerous other small changes to wording are documented in the track changes document.
\end{itemize}}

\section{Reviewer 1}

{\bf A generalized preferential sampling model}

Your random effects model doesn�t appear similar to \citet{RoyleBerliner1999} to me at all, and in fact this suggestion led to some confusion on my part about how these random effects were specified. The joint
specification of $\bfeta$ and $\bfdelta$, following \citet{RoyleBerliner1999} and assuming 0 mean, is:
\begin{equation}
  \left(
    \begin{array}{c}
    \bfeta \\
    \bfdelta
    \end{array}
   \right)
   \sim \textrm{Gaussian}
   \left(
     \left(
       \begin{array}{c}
       0 \\
       0
       \end{array}
     \right),
     \left(
       \begin{array}{cc}
         \bfSigma_\eta + \bfB \bfSigma_\delta \bfB^\prime & \bfB \bfSigma_\delta \\
         (\bfB \bfSigma_\delta)^\prime & \bfSigma_\delta
       \end{array}
     \right)
   \right),
\end{equation}
which has a very different covariance structure than the random effects model presented in Appendix S1.

\textcolor{blue}{There are several points to make here.  First, we never claimed that we were fitting the exact model proposed by \citet{RoyleBerliner1999} - we're not.  However, the covariance structure implied by the two model \emph{is} actually very similar.  For instance, if $\bfeta \sim \text{Gaussian}({\bf 0},\bfSigma_\eta)$ and $\bfdelta \sim \text{Gaussian}({\bf 0},\bfSigma_\delta)$, then
\begin{equation}
  \text{Cov}\left(
    \begin{array}{c}
    \bfeta + \bfB \bfdelta \\
    \bfdelta
    \end{array}
   \right)
   =
   \left(
   \begin{array}{cc}
      \bfSigma_\eta + \bfB \bfSigma_\delta \bfB^\prime & \bfB \bfSigma_\delta \\
         (\bfB \bfSigma_\delta)^\prime & \bfSigma_\delta
    \end{array}
   \right),
\end{equation}
as desired.  Second, it seems less important that the model be exactly the same as it does to provide a motivation (and citation) for how the $\bfB$ matrix can be specified.  We think both of these points were adequately accounted for in our initial submission, as we (a) described the \citet{RoyleBerliner1999} model as being ``similar" to the one we proposed here, and (b) used their paper as a reference for how the $\bfB$ matrix can be formulated and interpreted.
}

In your application, line 199, you report difficulties fitting a model with even modest structure on $\bfB$. Might a
joint model for $\bfeta$ and $\bfdelta$, as specified by \citet{RoyleBerliner1999}, allow you to specify a more complicated
model for $\bfB$?

\textcolor{blue}{
  As suggested above, we believe our model to be very similar to that proposed by \citet{RoyleBerliner1999}.  We're not seeing how drawing the random effects jointly would necessarily improve performance (we're also having trouble seeing how we could implement this in our chosen software).  Also, after further investigation of the \citet{RoyleBerliner1999} paper, we're not sure they ever actually fit a model with a more complicated relationship in $\bfB$.  In fact, their one example on ozone-temperature analysis uses the same simple diagonal parameterization we used here.  We've adjusted text in several places to reflect this. For instance, we state that ``However, to our knowledge the identifiability of higher order models (such as trend surfaces) has not been investigated" when talking about higher order preferential sampling models.
}

\textcolor{blue}{
  After correcting the error in our estimation model link function (see below), qualitative results have changed somewhat in the bearded seal example, with both the simple single parameter $b$ model and the trend surface model producing similar, vexing results. So, the question of identifiability may actually be one of estimability - i.e., how often do these methods produced reasonable estimates for real world datasets?  We know the single parameter $b$ model works $\approx 80\%$ of the time with simulated data; were we simply unlucky with the bearded seal data set? Or do other factors (e.g., overdispersion, line transects vs. point counts) affect the quality of inference?  In the discussion, we now suggest that identifiability and reliability of inference of such models be investigated in future research.
}

On line 139, you suggest �we often need to fix $\beta_0^* = 0.0$ . . . to permit parameter identification�. The word
``often" suggests you can sometimes include the intercept parameter when estimating random effects. Can
you provide more guidance about when you would need to fix the intercept at 0, and how a practitioner
would know they need to fit the intercept at 0?

\textcolor{blue}{
  Initially we were trying to fit ICAR type models which have a known parameter redundancy built in, such that the intercept is confounded with the spatial process \citep{RueHeld2005}.  However, the type of Gaussian process model we settled on and used in analysis does not suffer from this deficiency, but confusion resulted because somewhere along the way we mistakenly built in a multinomial logit link function into the estimation code for the $R_i$ (whose intercept is also not identifiable).  We have rectified this mistake in the present version, and all models for $R_i$ are now Bernoulli with a logit link, so that the intercept term is now identifiable.  We have rerun all analyses with this corrected configuration.
}

{\bf Notation}

I have difficulty following your notation at several points throughout the manuscript, which are outlined
below.

On line 41, you state $Z_i$ is the state variable of interest (e.g., occupancy or abundance), which I take will
be the notation you use throughout given the section header. Your specication for a count model on eqn
4 therefore becomes confusing. In particular, if $Z_i$ represents abundance, for reasonable values of $A_i$ and $p_i$ you could have random realizations $Y_i$ from a Poisson distribution that are greater than abundance, $Z_i$. If $Z_i$ does not represent abundance, what state process is represented here?

\textcolor{blue}{
  We have replaced the Poisson observation model with a binomial one in eqn 4 to make this clearer, and have indicated dependence on $Z_i$ through conditioning.  The reason that we had the Poisson in there originally is that the models we fit to data use the fact that the marginal distribution for $Y_i$ given the Poisson-Binomial mixture is itself a Poisson distribution.  However, we think it will be clearer to present the model hierarchically at this stage and then to describe our marginalization (something many ecologists may not care about) when describing software.  We have created a new ``Software" methods subsection which describes this in greater detail. Importantly, we describe how predictions are made; when detection probability is equal to 1.0, predictions are made using the formulation $\hat{N}_i = Y_i + (1-A_i)\exp(\hat{\mu}_i)$.  This framework incorporates a finite population correction and is similar to that employed by \citet{JohnsonEtAl2010} in the context of point process likelihood analysis of distance sampling data.  It also prevents predicted abundance in a cell from being less than the observed count.
}

On line 61, you indicate $\bfmu$ and $\bfnu$ represent log-scale abundance (associated with {\bf Z}) and the logit of the probability of sampling (associated with {\bf R}). For equation 5, however, you use $\bfmu_i$ to represent the mean value of a Gaussian process, which created some confusion. You might consider more generally allowing $\bfmu$ to be a function of covariates influencing the state variable of interest, and $\bfnu$ to be a function of covariates
influencing placement of sampling locations. You could then replace $\psi_i$ in equation 6 with $\nu_i$, which would make it immediately clear you were discussing factors influencing the probability of site placement.

\textcolor{blue}{
  We appreciate the reviewer's attempt to help us streamline notation and to make things clearer for the reader.  We have implemented the basic premise of this idea, although some of the particulars differ from the reviewer's suggestion.  Specifically we now describe $\bfmu$ and $\bfnu$ as indicating ``the link-scale process of interest and the logit of the probability of sampling, respectively.  As with $\bfmu$, this would mean $[Z_i] = f(\exp(\mu_i))$, whereas the distribution of sampling locations can be represented as $[R_i] = f(\text{logit}^{-1}(\nu_i))$, where $\mu_i$ and $\nu_i$ can be written as functions of regression coefficients, habitat covariates, and spatially autocorrelated random effects."  Then in the next section we use $\mu_i$ in eq. 5 and $\nu_i$ in eq. 6, although we retain $\xi_i$, since in the next section we also wish to write $\nu_i$ in terms of shared spatial effects.  This is all a little tricky, in that the \citet{DiggleEtAl2010} paper considers a continuous support model (whereas ours is discrete), and the structure is slightly different.  Admittedly, maintaining recognizable notation while keeping precise definitions is a challenge.
}

I understand the model you are specifying under the ``simulation study" subheading, but the notation is
different from that presented in the rest of the paper (i.e., use of ``=" vs. ``$\sim$"). I have no preference for
the notation used, but it seems like consistent notation would be more reader friendly. The same comment
holds true for the model presented under the subheading ``bearded seal count surveys".

\textcolor{blue}{
  We agree.  We've changed all distributions to brackets and have eliminated the ``$\sim$" notation.
}

Finally, your definition of $Z_i$ in the ``bearded seal count surveys" section seems inconsistent with the general
definition of $Z_i$ at the beginning of your manuscript. I think of state variables as presence / absence,
abundance, survival, etc. This definition for $Z_i$ seems more closely related to the expected value of a state
process.

\textcolor{blue}{
  We've rewritten the model here in terms of the Poisson-binomial hierarchical mixture.  In the following ``Software" section, we then indicate that we marginalize over $\bfZ$ when performing ML inference.
}

{\bf Other Comments}

Line 65: This sentence seems to discount the value of good study design, which is still important for model-
based estimation.

\textcolor{blue}{
We've altered text to be a little less cavalier, now stating that
``Design-based estimation requires that surveys rely on a pre-planned survey design selected probabilistically from an underlying sampling frame \citep{Cochran1977}.  Although randomization can increase legitimacy, model-based estimation is not formally subject to this requirement."
}

Line 77: It's unclear to me why the distinction between fine and coarse-scale preferential sampling is important. They both seem to represent the process of dependence between sampling locations and the process
of interest.

\textcolor{blue}{
As the size of the sampling unit decreases (with the limit being a point process model in continuous space) there is indeed little difference.  But practically, it seems likely that one could use a reasonable design to select sampling units, but still open themselves to bias if they target non-representative habitat within individual sampling units.  However, this point is well taken, and we don't think the distinction necessarily needs the prominence we've given it.  We now address this topic for the first time in the discussion.
}

Line 86: I do not see how the definition offered here differs from your initial description of preferential
sampling, unless the environmental covariate associated with the probability a location is selected is itself
independent from the state variable of interest, which seems unlikely in general.

\textcolor{blue}{
This is an extremely important point that we may have failed to get across.  Selecting sampling units based on a covariate related to abundance is just fine, \textit{provided} that the covariate is also used in the model for animal density/occurrence. In essence, we really only have preferential sampling if there the probability of sampling is related to the residuals (say $\hat{N_i} - N$) from a fitted SDM.  We've produced an additional figure (now fig. 1) that attempts to provide intuition about the distinction between sampling that is covariate-dependent and sampling that is preferential.
}

Line 186: I understand why you fit a relatively simple abundance model in this case. However, given the
difficulties you encountered in incorporating even modestly complex structure on B, how well do you suppose
your model will perform when you attempt to correct for nuisance processes?

\textcolor{blue}{
We don't know the answer to this question, although it seems like an important one.  In the discussion, we now suggest this as a topic for future research.
}

Line 195: You only describe 4 models below, not 6.

\textcolor{blue}{We now report results form all 6 models.}

Simulation study results: can you report bias for other parameters (e.g. slope parameters) in your model,
even if this is relegated to an appendix?

\textcolor{blue}{Yes, we have created an additional appendix table (S3) that summarizes bias in $b$ (the preferential sampling parameter), as well as bias in the species-habitat parameters.  We also provide convergence statistics in this appendix. Thanks for this suggestion; in particular, it helps show that preferential sampling is not necessarily a big deal if one is only interested in drawing inference on parameters informing species-habitat relationships.  %We note that the bias in the slope parameter of the species-habitat relationship %is likely because of the well-known confounding that occurs between regression %coefficients and spatially autocorrelated random effects in spatial models %\citep[e.g.][]{ReichEtAl2006,HodgesReich2010}.
}

Paragraph starting on line 219: I find this paragraph very confusing. In particular, it is unclear to me exactly
what you mean when you state you ``...refitted $M_{cov=1;b=0}$ to produce an estimate of apparent abundance
without this feature. Specifically, we refitted the model with 20 pseudo-absences in this portion of the study
area to better inform abundance-covariate relationships". Does Fig. 5 reflect this re-fitted model? Why not
fit all models in this manner for consistency?

\textcolor{blue}{Our initial thought was that inserting ``pseudo-absences" could possibly impact inference about the preferential sampling parameter because those locations were not actually sampled and shouldn't affect the $R_i$ model. So, we conducted model selection without pseudo-absences, but then inserted them when making predictions. In the revision, we've circumvented this issue by introducing depth as an additional covariate (with linear and quadratic effects). This seems to have remedied the issue with positive estimates of abundance in offshore areas.  So, no need to mess around with pseudo-absences anymore.}

Line 225: a model-averaged estimate of abundance?

\textcolor{blue}{After correcting the link function in present model runs, it no longer makes sense to present a model-averaged estimate.}

Table 1: I find these tables are easier to read if models are listed in decreasing order of $\Delta$AIC.

\textcolor{blue}{We now present models in order of decreasing predictive ability (as measured with cross-validated mean square prediction error)}

Fig. 3: In my version, the blue and green dashed lines do not show up in the legend.

\textcolor{blue}{Strange, this is not the case in our version.  Guessing it is an issue with PDF compilation on the MEE submission website.  We'll work with copy editors to ensure it looks right if it heads into production.}


\bibliography{master_bib}


\end{document}



























