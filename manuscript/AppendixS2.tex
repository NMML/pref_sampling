% besdoc.tex V1.0, 17 March 2011

\documentclass[times,mee,doublespace,]{besauth2}
%%\documentclass[times,mee,]{besauth}

\newcommand{\journalnamelc}{British Ecological Society}
\newcommand{\journalabb}{British Ecological Society}
\newcommand{\journalclassshort}{BES}
%%\newcommand{\journalname}{British Ecological Society}
\usepackage{epstopdf,comment}

\usepackage{moreverb}

\usepackage[colorlinks,bookmarksopen,bookmarksnumbered,citecolor=red,urlcolor=red]{hyperref}


\usepackage{lineno}

\newcommand\BibTeX{{\rmfamily B\kern-.05em \textsc{i\kern-.025em b}\kern-.08em
T\kern-.1667em\lower.7ex\hbox{E}\kern-.125emX}}

\bibpunct[; ]{(}{)}{;}{}{}{,}

\def\volumeyear{2015}
\def\VOC{VO}

\begin{document}

\begin{center} \bf {\large Confronting preferential sampling in wildlife surveys: diagnosis and model-based triage}

\rm
\vspace{0.7cm}
Paul B. Conn$^{1,*}$, James T. Thorson$^2$, Devin S. Johnson$^1$
\end{center}
\vspace{0.5cm}

\rm
\small


$^1$ National Marine Mammal Laboratory, Alaska Fisheries Science Center, NOAA National Marine Fisheries Service, 7600 Sand Point Way NE, Seattle, WA 98115 USA; $^2$ Fisheries Resource Assessment and Monitoring Division (FRAM), Northwest Fisheries Science Center, National Marine Fisheries Service (NMFS),
NOAA, 2725 Montlake Boulevard E, Seattle, WA 98112, USA

$^*$paul.conn@noaa.gov

\large
\bigskip
\centerline{Appendix S2: Specific details on parameter estimation in bearded seal example}
\bigskip
\small

\linenumbers

\def\VAR{{\rm Var}\,}
\def\COV{{\rm Cov}\,}
\def\Prob{{\rm P}\,}

\begin{flushleft}

We attempted to fit four models that estimated preferential sampling parameter(s) to bearded seal data.  The models differed by whether or not habitat and landscape covariates were modeled, and how the preferential sampling effect was modeled.  Specifically, when the probability $R_i$ that a location $i$ is selected for sampling is modeled as
\begin{eqnarray}
 \label{eq:R}
  R_i & \sim & \text{Bernoulli}(\textrm{logit}^{-1}(\nu_i)),
\end{eqnarray}
where
\begin{eqnarray}
  \label{eq:nu}
  \nu_i & = & \beta_0^* + {\bf x}_i^* \boldsymbol{\beta}^*+\eta_i+{\bf B} \delta_i,
\label{eq:nu}
\end{eqnarray}
we considered two formulations for \textbf{B}.  In the first, it was simply set to $b \textbf{I}$, where $\textbf{I}$ is an $(S \times S)$ identity matrix and $b$ is an estimated parameter.  In the second, we implemented a trend surface model by writing $\bf{B}$ as
\begin{eqnarray}
  \label{eq:B}
  \textbf{B} & = & {\bf C}{\bf b} \textbf{I}.
\end{eqnarray}
Here, ${\bf b}$ is a column vector consisting of three estimated parameters, $b_0$, $b_1$, and $b_2$, and ${\bf C}$ is an $(S \times 3)$ design matrix constructed as follows:
\begin{eqnarray}
  \textbf{C} & = & \left[
    \begin{array}{ccc}
    1 & northing_1 & easting_1 \\
    1 & northing_2 & easting_2 \\
    \vdots & \vdots & \vdots \\
    1 & northing_S & easting_S
    \end{array}
  \right].
\end{eqnarray}
The northing and easting values are simply the $y$ and $x$ coordinates associated with each grid cell centroid on the particular geographical projection of the earth's surface that we were using (in our case, EASE Grid 2.0).

\hspace{0.5in} On the surface, there did not appear to be any numerical errors when fitting preferential sampling models to bearded seal data.  The Hessian was invertible, and incorporation of the preferential sampling effects resulted in much lower AIC scores than the other models (AIC scores were close to 100 points better than the highest ranked AIC model omitting preferential sampling effects).  However, when examining estimates graphically, it was clear that the preferential sampling models suffered from some aliasing issues (Fig. \ref{fig:bearded_RE_maps}).  Specifically, most variation in the probability of sampling was passed on from the $\boldsymbol{\eta}$ random effect to the shared random effect $\boldsymbol{\delta}$ such that it was difficult to disentangle the sampling process from the abundance process.  To our mind, this was a clear sign that the model was overparameterized, leading to poor estimation.

\raggedbottom



\bibliographystyle{bes}
\bibliography{master_bib}

\pagebreak

\begin{figure*} %[ht]
\begin{center}
\includegraphics[width=170mm]{bearded_RE_maps.pdf}
\caption{Estimates of spatial random effects for two models fitted to bearded seal count data.  Estimates from preferential sampling model $M_{cov=1,b=1}$ are provided in the first column of plots, while estimates from a second model with a trend surface specification for the preferential sampling parameter appear in the second column (this model also included covariate effects).  In the first column, the $\boldsymbol{\eta}$ random effects specific to the sampling probability model appear well estimated and to indicate a higher probability of sampling where transects were flown (as intended).  However, under the trend surface specification, these random effects are estimated to be close to zero, and the increased probability of sampling is passed on to the shared $\boldsymbol{\delta}$ random effects.}
\label{fig:bearded_RE_maps}
\end{center}
\end{figure*}

\end{flushleft}
\end{document}

%http://www.plosone.org/article/info:doi/10.1371/journal.pone.0036527
